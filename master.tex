\documentclass[a4paper,10pt,swedish]{memoir}

\usepackage{listings}
\usepackage{minted}
%\usepackage{fancyhdr}
\usepackage{xunicode}
\usepackage{xltxtra}
\DeclareUTFcharacter[\UTFencname]{x00A0}{\nobreakspace}
\usepackage{polyglossia}
\usepackage[pdfusetitle,bookmarks=true,
 bookmarksnumbered=true,bookmarksopen=false,
 breaklinks=false,pdfborder={0 0 0},backref=false,
 colorlinks=false]{hyperref}
\usepackage{color}
\usepackage{fontspec}
\usepackage{amsmath}
\usepackage{amsfonts}
\usepackage{mathtools}
\usepackage{tikz}

\setmainfont[Mapping=tex-text,Ligatures={Common},Numbers={OldStyle},Contextuals={Alternate},Scale=MatchLowercase]{Garamond
  Premier Pro} \setsansfont[Scale=MatchLowercase]{Futura Std}
\setmonofont[Scale=MatchLowercase]{PragmataPro}

\usemintedstyle{friendly}
\renewcommand{\theFancyVerbLine}{
  \textcolor[rgb]{0.5,0.5,0.5}{\scriptsize
    \arabic{FancyVerbLine}}}

\newminted{matlab}{}
\newmint{matlab}{}
\newminted{text}{}

%\pagestyle{fancy}
%\lhead{DN1241 - Numerical Methods}
%\renewcommand{\headrulewidth}{0pt}

\setdefaultlanguage{swedish}

\chapterstyle{section}
\setcounter{tocdepth}{3}
\setlength{\parskip}{\smallskipamount}
\setlength{\parindent}{0pt}
\setlength{\intextsep}{0pt}

\numberwithin{equation}{subsection}

\title{Anteckningar - DN1241 - Numerical Methods}
\author{Källa: Fabian Bergmark, Typsättning: David Flemström}
\date{\today}

\begin{document}
\maketitle

\tableofcontents

\chapter{Föreläsning 1}
\section{Lokal linjärisering}

Nästan alla funktioner kan lokalt ersättas med plana linjer.

Grundidé:

\begin{itemize}
\item Ekvationslösning
\item Derivataskattning
\item Lösa differentialekvationer
\item Integraler
\item Felanalys
\end{itemize}

Vi arbetar endast med funktioner som är lokalt linjäriserbara.

\section{Matlab}

\begin{matlabcode}
x = -5:0.1:10;
y = x - 4 * sin(2 * x) - 3;
plot(x, y);
subplot(2, 2, 1);
diary linekv.d;
diary off;
\end{matlabcode}

\subsection{Tilldelning}

\begin{matlabcode}
a = 3 % (= 3 + 0*i)
z = 5 + 2;
x = [1; 2; 7]; % För att referera till element: x(2) => 2
A = [1 2; 5 9];
A = zeros(10, 10); % A(5, 1) => 0
\end{matlabcode}

\subsection{Utskrift}

Använd inte semikolon.

\begin{matlabcode}
disp(x); % skriver endast ut värdet
format long;
\end{matlabcode}

\subsection{Inmatning}
\matlab:r = input('Ge räntesatsen');:

\subsection{Programkonstruktion}
Repetitionssatser

\begin{matlabcode}
for i = 1:5
    disp('*');
end

i = 0;
while (i < 5)
    i = i + 1;
    disp('*');
end
\end{matlabcode}

\subsection{Matematiska operatorer}

\begin{matlabcode}
.*, ./, .^
sin(), exp(), sqrt()
\end{matlabcode}

\subsection*{Problem}

\begin{align}
y&=x-4\cdot\sin\left(2x\right)-3=0\\
f(x)&=0\\
x&\approx 1.732 \qquad\text{(Enligt bild)}
\end{align}

Vi vill ha ett fel $<10^{-16}$

\section{Newtons metod}

$x_i$ är en approximativ lösning.

\begin{align}
y-y_i &= f'(x_i)(x-x_i) &\text{Tangentens ekvation}\\
y&=f(x_i)+f'(x_i)(x-x_i) &\text{Linjens ekvation}
\end{align}

Vi söker räta linjens skärning med x-axeln; sätt $y=0$.

\begin{align}
0&=f(x_i)+f'(x_i)(x-x_i)
x_{i+1}=x_i-\frac{f(x_i)}{f'(x_i)}
\end{align}

Matlab:

\begin{matlabcode}
x = 1.732;
f = x - 4 * sin(2 * x) - 3;
fp = 1 - 8 * cos(2 * x); % = f'(x)
h = f / fp
x = x - h
\end{matlabcode}

\section{Felbegrepp}

$\tilde{x}$ approximerar $x$. Ex $\tilde{x}=3.14, x=\pi$

\begin{align}
\tilde{x}-x\qquad&\text{absolutfelet} \qquad e_x \\
\frac{\tilde{x}-x}{x}\qquad&\text{relativa felet}\qquad r_x
\end{align}

\section{Felgränser}

\begin{align}
|e_x| &\leq E_x \\
|r_x| &\leq R_x
\end{align}

Ex.

\begin{align}
g&=9.81\pm 0.005 \\
\tilde{g}&=9.81 \\
E_x&=0.005
\end{align}

\subsection{Exempel}

\begin{align}
f(x)&=e^x\\
f'(x)&=e^x
\end{align}

\subsubsection{Analys}

Vilka fel finns i approximationen?

\begin{description}
\item[Trunkeringsfel]
  kan analyseras m.h.a. Tailorutv.
  \begin{align}
    f(a+h) &= f(a) + h f'(a) + \frac{1}{2}h^2 f''(a) + \ldots +
    \frac{1}{k!} h^kf^{(k)}(a) \\
    \Delta h &= \frac{1}{h}\left(h f'(a) + \frac{1}{2}h^2 f''(a) +
    O\left(h^3\right)\right) \\
    &= f'(a)+\frac{1}{2}h f''(a) + O(h^2)
  \end{align}
  Trunkeringsfelet är $\Delta h - f'(a)=\underbrace{\frac{1}{2}h
    f''(a)}_{\mathclap{\text{Dominerande delen av felet}}}+O(h^2)$

  Således är trunkeringsfelet $\propto h$, noggrannhetsordning 1.
\item[Precisionsfel] är fel pga. fel i funktionsvärdena.

Låt $f(x)$ representeras av $\tilde{f}(x)$ i datorn. Pga. ändlig
precision så gäller att $f(x)\neq\tilde{f}(x)$. Låt
$\tilde{f}(x)=f(x)+\epsilon_{f(x)}$. Antag att $|\epsilon_{f(x)}|\leq
E_f, \forall x \in D_f$.

\begin{equation}
\tilde{\Delta}_h = \frac{\tilde{f}(a+h)-\tilde{f}(a)}{h} =
\frac{f(a+h) + \epsilon_{f(a+h)} - f(a) - \epsilon_{f(a)}}{h}
\end{equation}

\begin{equation}
\Delta_h + \frac{\epsilon_{f(a+h)}-\epsilon_{f(a)}}{h} \qquad
\text{så} \qquad |\epsilon_h| \leq \frac{2 E_f}{h}
\end{equation}
\end{description}

\subsubsection{Matlab}

\begin{matlabcode}
f = inline('exp(x)', 'x');
h = 1; a = 1; ea = f(a); format long; format compact;

for i = 1:8
    fp = (f(a+h) - f(a) / h);
    disp([h fp]);
    h = h / 10;
end
\end{matlabcode}

\section{Egendefinierade funktioner}

Definieras i egen fil \verb:f2a.m:.

\begin{matlabcode}
function r = f2a(x)
    % r är returvärde
    % f2a funktionsnamn
    % x argument
    r = x.^2 - 2;

test = f2a([1; 2]) % => [-1; 2]
\end{matlabcode}

\newpage

\chapter{Föreläsning 2}

\subsection{Exempel}

En person märker ut avstånd genom att stega.

\begin{figure}[h]
\vspace{1cm}
\centering
\begin{tikzpicture}
  \draw (0,0) -- (10,0);
  \foreach \x/\xtext in {0/ ,6/$x_1$,8/$x_2$,10/$100$}
    \draw (\x,0pt) -- (\x,-3pt) node[below] {\xtext};
\end{tikzpicture}
\caption{Två utmärkta punkter längs en tallinje.}
\end{figure}

\begin{equation}
x_1\approx 60 \qquad
x_2\approx 80 \qquad
x_2-x_1\approx 20 \qquad
100 - x_2\approx17 \qquad
100 - x_1 \approx38
\end{equation}

Vi har ett system:

\begin{align}
A\vec{x}&=\vec{y} \qquad
A &=
\left[ \begin{array}{cc}
1 & 0 \\
-1 & 1 \\
0 & 1 \\
1 & 0
\end{array}\right] \\
\vec{x} &= \left[\begin{array}{c}
x_1 \\
x_2
\end{array}\right] \qquad
\vec{y} &= \left[\begin{array}{c}
60 \\
20 \\
83 \\
62
\end{array}\right]
\end{align}

4 ekv, 2 obekanta; överbestämt ekvationssystem.

\subsubsection{Matlab}

\begin{matlabcode}
A = [1 0; -1 1; 0 1; 1 0];
y = [60; 20; 83; 62];
x = A \ y;
\end{matlabcode}

Vi vet att $A\vec{x}=\vec{y}$. Vi kan bilda residualvektorn
$\vec{r}=\vec{y}-A\vec{x}$

\subsubsection{Geometrisk tolkning}

Vad avses med bästa $\vec{x}$? Jo den vektor som gör $\vec{r}$ så
liten som möjligt. Liten avser den euklidiska normen.

\begin{equation}
||\vec{r}||_2 = \sqrt{r_1^2 + r_2^2 + \ldots + r_m^2}
\end{equation}

Vi kan använda residualvektorn för felkontroll.

\begin{align}
A^TA \vec{x} &= A^T\vec{y} \\
0 &= A^T(\vec{y}-A\vec{x}) = A^T \vec{r}
\end{align}

Vi kan således kontrollera våra beräkningar med $A^T\vec{r} = 0$.

\section{Linjär modellanpassning}

Modell: Rät linje $p(t) = C_1 + C_2 t$

Data: \[\left[\begin{array}{cc}40&55.3\\45&71.9\\50&92.5\\55&118.0\\60&149.9\end{array}\right]\]

\begin{align}
C_1 + C_2 \cdot 40 &= 55.3 \\
C_1 + C_2 \cdot 45 &= 71.9 \\
&\vdots
\end{align}

\begin{equation}
A\vec{c}\approx \vec{p}
\end{equation}

Vi ställer upp och löser

\begin{align}
A^TA\vec{c}&=A^T\vec{p}
\vec{c}=(A^TA)^{-1}A^T\vec{p}
\end{align}

\subsection{Utvigningar}

Modell: $q(t) = C_1 + C_2 t + C_3 t^2$

Matlab:

\begin{matlabcode}
% Indata
T = 40:5:60;
p = [55.3 ...];

% Formulera och lös systemet.
A = [ones(size(T))' T' T.^2'];
c = A \ p;
r = p - A * c;
TP = 40:0.1:60;
PP = c(1) + c(2)*TP + c(3)*TP.^2;

plot(TP, PP, 'o');
\end{matlabcode}

\subsection{Problem}

Givet mätvärden $(t_1, y_1), i=1,2,\ldots,m$ och en linjär funktion
(modell) $F=C_1 \Phi_1(t) + C_2\Phi_2(t) + \ldots + C_n\Phi_n(t), m >
n$ med okända koefficienter $\vec{C}$.

Bestäm bästa $\vec{C}$.

Anm: $\vec{\Phi}(t)$ är basfunktionerna, kända.

\begin{equation}
\vec{\Phi}_1 = \left[\begin{array}{c}
\Phi_1(t_1) \\
\Phi_1(t_2) \\
\vdots \\
\Phi_1(t_m) \\
\end{array}\right] \in \mathbb{R}^m
\end{equation}

Anm: om $m=n$ får vi interpolation.

Exempel på basfunktioner:

\begin{align}
\Phi(t) &= t^{i-1}, i = 1,2,\ldots,n \qquad \text{Polynom bas} \\
\Phi_1(t) &= 1, \Phi_2(t) = t-k, \Phi_3(t) = (t-k)^2, k = \sum_{i=1}^{m}\frac{t_i}{m} \\
\Phi(t) &= \left[1, cos(t), sin(t), cos(2t), sin(2t)\right]
\end{align}

\subsubsection{Arbetsgång}

Sätt in $(t_i, y_i), i=1\ldots m$ i modellen.

\begin{align}
C_1\Phi_1(t_1) + C_2\Phi_2(t_1) + \ldots + C_n\Phi_n(t_1) &\approx y_1 \\
\vdots &\vdots \\
C_1\Phi_1(t_m) + C_2\Phi_2(t_m) + \ldots + C_n\Phi_n(t_m) &\approx y_m
\end{align}
\begin{equation}
A\cdot \vec{C} \approx \vec{y} \qquad A=\left[\Phi_1,\Phi_2, \ldots,\Phi_n\right]
\end{equation}

Lös med normalekvationerna.

\begin{equation}
\vec{r} = \vec{r} - A\vec{C}
\end{equation}

\begin{align*}
\Phi(\vec{C}) &= ||\vec{r}||_2^2 = \vec{r}^T\vec{r} = (\vec{y}-A\vec{C})^T(\vec{y}-A\vec{C}) \\
&= (\vec{y}^T-\vec{C}^TA^T)(\vec{y}-A\vec{C})=\vec{y}^T\vec{y}-\vec{y}^TA\vec{C} - \vec{C}^TA^T\vec{y} + \vec{C}^TA^TA\vec{C} \\
&= \vec{C}^TA^TA\vec{C}-2\vec{C}^TA^T\vec{y} + \vec{y}^T\vec{y}
\end{align*}
\begin{align}
\nabla\vec{C}\Phi(\vec{C})&=0 \qquad \text{ger minimit} \\
\nabla\vec{C}\Phi(\vec{C})&=0=2A^TA\vec{C}-2A^T\vec{y} \\
A^TA\vec{C}&=A^T\vec{y}
\end{align}

\begin{table}[h]
\centering
\begin{tabular}{l l}
x & y \\ \hline
15 & 15 \\
16 & 8 \\
17 & 5 \\
18 & 3 \\
19 & 3
\end{tabular}
\caption{Exempelmätvärden}
\end{table}

\begin{align}
F(x)&=a_1 + a_2 x + a_3x^2 \qquad &\text{(ger stora tal)} \\
F(x)&=C_1 + C_2(x-m) + C_3(x-m)^2, m=17 \qquad &\text{centrering}
\end{align}

\section{Vektor och matrisnormer}

\begin{align}
||\vec{x}||_2 &= \sqrt{x_1^2 + x_2^2 + \ldots + x_n^2} = \sqrt{\vec{x}^T\cdot\vec{x}} \\
||\vec{x}||_p = \sqrt[p]{x_1^p+x_2^p+\ldots+x^p_n}
\end{align}

En norm är ett icke-negativt mätetal för vektorer.

\subsection{Axiom}

\begin{align}
||\vec{x}|| & > 0 \forall \vec{x} \neq \vec{0} \\
||\vec{x}|| & = 0 \qquad \text{omm} \qquad \vec{x}=\vec{0} \\
||\alpha\vec{x}||&=|\alpha|\cdot||\vec{x}||, \qquad\text{$\alpha$ skalär} \\
||\vec{x}+\vec{y}|| &\leq ||\vec{x}||+||\vec{y}|| \\
p &= \infty \qquad \text{maxnormen}\\
||\vec{x}||_\infty &= \text{max}(\vec{x})
\end{align}

\chapter{Föreläsning 3}

% TODO

\end{document}
